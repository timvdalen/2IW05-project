\subsection{Class diagram}
	\begin{figure}[H]
		\centering
		\begin{tikzpicture}
			%Ignore the {x,y} values while working on this, we'll change that later
			\begin{class}[text width=4cm]{User}{0,5}
			\attribute{- id : integer}
			\attribute{- name : String}
			\attribute{- password : String}
			\attribute{- emailAdress: String}
			\end{class}
		
			\begin{class}[text width=6cm]{Vote}{0,10}
			\attribute{- dateTimeSlot : DateTimeSlot }
			\operation{+ userAgreed(user: User)}
			\end{class}

			\begin{class}[text width=4cm]{DateTimeSlot}{0,14}
			\attribute{- startDate : Date}
			\attribute{- endDate : Date}
			\end{class}

			\begin{class}{Poll}{9,13}
			\attribute{- hidden : Bool}
			\attribute{- limit: Integer}
			\attribute{- closed: Bool}
			\attribute{- options: DateTimeSlot[1..*]}
			\operation{+ removeVote(vote : Vote)}
			\operation{+ addVote(vote : Vote)}
			\operation{+ changeVote(vote : Vote, changedVote: Vote)}
			\operation{+ sendNotificationEmail()}
			\operation{+ addTimeSlot(dateTimeSlot : DateTimeSlot)}
			\end{class}

			\begin{class}[text width=6cm]{Event}{9,3}
			\attribute{- id : integer}
			\attribute{- name : String}
			\attribute{- description: String}
			\attribute{- participants: User[*]}
			\attribute{- optionalMessage: String}
			\operation{+ sendInvitations()}
			\operation{+ openPoll()}
			\operation{+ closePoll()}
			\operation{+ addUser(user: User)}
			\operation{+ removeUser(user: User)}		
			\operation{+ addResponse(response: Response)}
			\operation{+ changeResponse(oldResponse : Response, newResponse: Response)}

			\end{class}

			\begin{class}[text width=3cm]{Response}{0,0}
			\attribute{- present : Bool}
			\end{class}


			%Syntax: \aggregation{ClassA}{varOfClassA}{MultofA}{ClassB}{MultofB}{varOfClassB}
			%I'm pretty sure not all are a two-way association
			\association{Response}{responses}{0..*}{User}{1}{user}
			\association{Response}{responses}{0..*}{Event}{1}{event}
			\association{DateTimeSlot}{timeslot}{1}{Vote}{1}{vote}
			\association{Vote}{castVotes}{0..*}{User}{0..*}{usersAgreed}
			\association{Vote}{votes}{0..*}{Poll}{1}{poll}
			\association{User}{organizer}{1}{Event}{organizedEvents}{0..*}
			\composition{Event}{poll}{1}{Poll}
		\end{tikzpicture}
		\caption{High-level entities in the system}
		\label{fig:class-diagram}
	\end{figure}

	For most private attributes we have left out getters and setters because they are implied. For some attributes however, like the responses attribute of Event and the votes attribute of Poll, for which elements can change, we have supplied some getters, setters and changers.

	Classes represent real-world entities as described in section~\ref{sec:analysisdiagram}.

	\emph{Note: We assume there exists a class called \textnormal{Date} that represents a date and a time.}

\subsection{Message Sequence Diagrams}
	\subsubsection{Create meeting schedule}i
		\begin{figure}[H]
			\centering
			\begin{msc}{Create meeting schedule}
				\declinst{org}{User}{Organizer}
				\dummyinst{evt}
				\dummyinst{poll}
				\declinst{email}{Email}{}
				\declinst{inv}{User}{Invitee}

				\create{new}{org}{evt}{Event}{}
				\nextlevel
				\nextlevel
				\create{new}{evt}{poll}{Poll}{}
				\nextlevel

				\mess{setName()}{org}{evt}
				\nextlevel
				\mess{setDescription()}{org}{evt}
				\nextlevel
				\mess{addTimeSlot()}{org}{evt}
				\nextlevel
				\mess{addTimeSlot()}{evt}{poll}
				\nextlevel
				\mess{addTimeSlot()}{org}{evt}
				\nextlevel
				\mess{addTimeSlot()}{evt}{poll}
				\nextlevel
				\nextlevel

				\mess{addUser()}{org}{evt}
				\nextlevel
				\mess{addUser()}{org}{evt}
				\nextlevel
				\nextlevel

				\mess{setHidden()}{org}{evt}
				\nextlevel
				\mess{setHidden()}{evt}{poll}
				\nextlevel
				\nextlevel

				\mess{openPoll()}{org}{evt}
				\nextlevel
				\regionstart{coregion}{evt}
				\nextlevel
				\mess{Send confirmation}{evt}{email}
				\nextlevel
				\mess{Send confirmation}{email}{org}
				\nextlevel
				\mess{sendInvitation()}{evt}{email}
				\nextlevel
				\mess{sendInvitation()}{email}{inv}
				\nextlevel
				\regionend{evt}
				\nextlevel
			\end{msc}
			\caption{Create meeting schedule}
			\label{msc:createmeeting}
		\end{figure}

		The procedure modelled above will be executed once the User, the Organizer of the Event, clicks the link indicating that he or she wants to create a new Event. 
		To do so the User Organizer needs to enter a name, description and possible timeslots. These time slots will be entered into the Poll. After entering this general information the Organizer needs to add other Users to the event, to be invited. All that is left is for the Organizer to choose wether to make the Poll visible to the other users. Now that all options have been set the Organizer needs to open the Poll.
		After all event information has been received, messages will be sent through the Email service to all Users involved. A confirmation message will be sent to the 	Organizer and Invitations will be sent to all invited Users.

	\subsubsection{Respond to meeting invitation}
		\begin{figure}[H]
			\centering
			\begin{msc}{Respond to meeting invitation}
				\declinst{usr}{User}{Invitee}
				\dummyinst{resp}
				\declinst{evt}{Event}{}
				\declinst{poll}{Poll}{}
				\declinst{email}{Email}{}
				\declinst{org}{User}{Organizer}

				\create{new}{usr}{resp}{Response}
				\nextlevel
				\nextlevel
				\nextlevel
				\mess{setPresent()}{usr}{resp}
				\nextlevel
				\nextlevel

				\mess{addResponse(this, response)}{usr}{evt}
				\nextlevel
				\nextlevel

				\mess*{options}{poll}{evt}
				\mess*{Show DateTimeSlot options}{evt}{usr}
				\nextlevel
				\mess{Vote on options}{usr}{evt}
				\mess{Vote}{evt}{poll}
				\nextlevel
				\mess{Notification}{evt}{email}
				\mess{Send notification}{email}{org}
				\nextlevel
				\nextlevel
				
				\mess{Show stats}{evt}{usr}
				\nextlevel
			\end{msc}
			\caption{Respond to meeting invitation}
			\label{msc:respondinvite}
		\end{figure}

		This MSC models a run of the system for the usecase "Respond to meeting invitation", which describes a case where a user receives an invitation to attend an event. In our model, we assume that this can only happen via email, but one could imagine a system where already existing users receive a notification op the homepage of the website.

		The user creates a new Response, containing 'true' for his presence. This Response is added to the Event the user was invited for. The Event reads the options from its Poll and then shows them to the User. The user submits a Vote. The Event saves this in the Poll.
		It then notifies the email service which will send an email to the organizer.
		The Event will then show the statistics (number of votes for each possible DateTimeSlot) to the User.

	\subsubsection{Edit meeting response}
		\begin{figure}[H]
			\centering
			\begin{msc}{Respond to meeting invitation}
				\declinst{usr}{User}{Invitee}
				\declinst{resp}{Repsonse}{}
				\declinst{evt}{Event}{}
				\declinst{poll}{Poll}{}
				\declinst{email}{Email}{}
				\declinst{org}{User}{Organizer}

				\mess{setPresent()}{usr}{resp}
				\nextlevel
				\nextlevel

				\mess{changeResponse(this, response)}{usr}{evt}
				\nextlevel
				\nextlevel

				\mess*{options}{poll}{evt}
				\mess*{Show DateTimeSlot options}{evt}{usr}
				\nextlevel
				\mess*{previous votes}{poll}{evt}
				\mess*{Show previous votes}{evt}{usr}
				\nextlevel
				\mess{Vote on options}{usr}{evt}
				\mess{Vote}{evt}{poll}
				\nextlevel
				\mess{Notification}{evt}{email}
				\mess{Send notification}{email}{org}
				\nextlevel
				\nextlevel
				
				\mess{Show stats}{evt}{usr}
				\nextlevel
			\end{msc}
			\caption{Respond to meeting invitation}
			\label{msc:respondinvite}
		\end{figure}

		This message sequence diagram models the "Edit meeting response" use case. A user can change his present for the event. He can confirm that he is present or he can change it in a later stage when he decide he will not come to the event. When he is present he will vote when the event will be planned. If the votes are not hidden the user can see if the others are present.\\\\
		The first step is when the user set his present, the user decide he is present or not. This will be send to the response class so the response class if the user is coming or not. The user can change this present status anytime ( if the event is not cancelled or closed) then the present status of the corresponding respone will be changed to the desired present status. If the user is present then vote options of the timestamps will be send to the user. The user can also see the votes from other people when the poll is not hidden. After this the user can send his votes. And at least the system returns the stats of the event to the user.

	\subsubsection{Administrate event}
		\begin{figure}[H]
			\centering
			\begin{msc}{Administrate event}
				\declinst{org}{User}{Organizer}
				\declinst{evt}{Event}{}

				\mess{Show event overview}{evt}{org}
				\nextlevel
				\mess{Show options}{evt}{org}
				\nextlevel
				\mess{Change option/delete/view log}{org}{evt}
				\nextlevel
				\mess*{}{evt}{org}
			\end{msc}
			\caption{Administrate event}
			\label{msc:adminevent}
		\end{figure}

	\subsubsection{Close event}
		\begin{figure}[H]
			\centering
			\begin{msc}{Close an event}
				\declinst{org}{User}{Organizer}
				\declinst{evt}{Event}{}
				\declinst{poll}{Poll}{}
				\declinst{email}{Email}{}
				\declinst{inv}{User}{Invitee}

				\mess{Are you sure?}{evt}{org}
				\nextlevel
				\mess{Yes}{org}{evt}
				\nextlevel
				\nextlevel

				\mess{Poll results}{evt}{org}
				\nextlevel
				\mess{Select final DateTimeSlot}{org}{evt}
				\nextlevel
				\mess{Enter optional closing message}{org}{evt}
				\nextlevel
				\mess{Confirm email}{evt}{email}
				\nextlevel
				\mess{setClosed()}{evt}{poll}
				\regionstart{coregion}{email}
				\nextlevel
				\mess{Confirm email}{email}{inv}
				\nextlevel
				\mess{Confirm email}{email}{org}
				\nextlevel
				\regionend{email}
			\end{msc}
			\caption{Close an event}
			\label{msc:closeevent}
		\end{figure}
