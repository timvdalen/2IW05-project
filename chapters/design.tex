\subsection{Class diagram}
	\begin{figure}[H]
		\centering
		\begin{tikzpicture}
			\begin{class}{User}{0,3}
			\end{class}
		
			\begin{class}{Vote}{0,6}
			\end{class}

			\begin{class}{DateTimeSlot}{8,6}
			\end{class}


			\begin{class}{Event}{8,3}
			\end{class}

			\begin{class}{Response}{4,0}
			\end{class}

			\association{Response}{user}{1}{User}{0..*}{responses}
			\association{Response}{event}{1}{Event}{0..*}{reponses}
			\association{DateTimeSlot}{timeslot}{1}{Vote}{vote}{1}
			\association{Vote}{user}{1}{User}{1}{event}
			\association{User}{organizedEvents}{0..*}{Event}{1}{organizer}
		\end{tikzpicture}
		\caption{High-level entities in the system}
		\label{fig:class-diagram}
	\end{figure}

\subsection{Message Sequence Diagrams}
	\subsubsection{One}
		This MSC describes...
%%\begin{figure}[H]
%%%\centering
%%%\begin{msc}{One}
%%%%\declinst{comp}{Component 1}{}
%%%%\declinst{compt}{Component 2}{}
%%%
%%%%\mess{Message}{comp}{compt}
%%%\end{msc}
%%%\caption{One}
%%%\label{msc:one}
%%\end{figure}
